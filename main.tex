\documentclass[12pt]{article}
\usepackage[utf8]{inputenc}
\usepackage[russian]{babel}
\usepackage{amsmath,amssymb}
\usepackage{graphics}
\usepackage{pbox}
\usepackage[x11names]{xcolor}
\definecolor{brightmaroon}{rgb}{0.76, 0.13, 0.28}
\definecolor{royalazure}{rgb}{0.0, 0.22, 0.66}
\usepackage[colorlinks=true,linkcolor=royalazure]{hyperref}
\usepackage{tikz, tkz-fct, pgfplots}
\usetikzlibrary{arrows}
\usepackage{geometry}
\geometry{
	a4paper,
	total={170mm,257mm},
	left=20mm,
	top=20mm
} 
\usepackage[labelsep=period]{caption}

% ----------------- Commands ----------------- 
\newcommand{\eps}{\varepsilon}
\newcommand\tline[2]{$\underset{\text{#1}}{\text{\underline{\hspace{#2}}}}$}

% ----------------- Set graphics path ----------------- 
\graphicspath{{img/}}

\begin{document}
\newpage 
\pagestyle{empty}
\centerline{\large Министерство науки и высшего образования}	
\centerline{\large Федеральное государственное бюджетное образовательное}
\centerline{\large учреждение высшего образования}
\centerline{\large ``Московский государственный технический университет}
\centerline{\large имени Н.Э. Баумана}
\centerline{\large (национальный исследовательский университет)''}
\centerline{\large (МГТУ им. Н.Э. Баумана)}
\hrule
\vspace{0.5cm}
\begin{figure}[h]
\center
\includegraphics[height=0.35\linewidth]{bmstu-logo-small.png}
\end{figure}
\begin{center}
	\large	
	\begin{tabular}{c}
		Факультет ``Фундаментальные науки'' \\
		Кафедра ``Высшая математика''		
	\end{tabular}
\end{center}
\vspace{0.5cm}
\begin{center}
	\LARGE \bf	
	\begin{tabular}{c}
		\textsc{Отчёт} \\
		по учебной практике \\
		за 1 семестр 2020---2021 гг.
	\end{tabular}
\end{center}
\vspace{0.5cm}
\begin{center}
	\large
	\begin{tabular}{p{5.3cm}ll}
		\pbox{5.45cm}{
			Руководитель практики,\\
			ст. преп. кафедры ФН1} 	& \tline{\it(подпись)}{5cm} & Кравченко О.В. \\[0.5cm]
		студент группы ФН1--11 		& \tline{\it(подпись)}{5cm} & Васильев Егор Ринатович.
	\end{tabular}
\end{center}
\vfill
\begin{center}
	\large	
	\begin{tabular}{c}
		Москва, \\
		2020 г.
	\end{tabular}
\end{center}

\newpage	
\tableofcontents

\newpage
\section{Цели и задачи практики}	
\subsection{Цели}
--- развитие компетенций, способствующих успешному освоению материала бакалавриата и необходимых в будущей профессиональной деятельности.

\subsection{Задачи}
\begin{enumerate}
\item Знакомство с программными средствами, необходимыми в будущей профессиональной деятельности.
\item Развитие умения поиска необходимой информации в специальной литературе и других источниках.
\item Развитие навыков составления отчётов и презентации результатов.
\end{enumerate}

\subsection{Индивидуальное задание}	
\begin{enumerate}
\item Изучить способы отображения математической информации в системе вёртски \LaTeX.
\item Изучить возможности  системы контроля версий \textsf{Git}.
\item Научиться верстать математические тексты, содержащие формулы и графики в системе \LaTeX.
Для этого, выполнить установку свободно распространяемого дистрибутива \textsf{TeXLive} и оболочки \textsf{TeXStudio}.
\item Оформить в системе \LaTeX типовые расчёты по курсе математического анализа согласно своему варианту.
\item Создать аккаунт на онлайн ресурсе \textsf{GitHub} и загрузить исходные \textsf{tex}--файлы 
и результат компиляции в формате \textsf{pdf}.
\end{enumerate} 

\newpage
\section{Отчёт}
Актуальность темы продиктована необходимостью владеть системой вёрстки \LaTeX и средой вёрстки \textsf{TeXStudio} для
отображения текста, формул и графиков. Полученные в ходе практики навыки могут быть применены при написании
курсовых проектов и дипломной работы, а также в дальнейшей профессиональной деятельности.

Ситема вёрстки \LaTeX содержит большое количество инструментов (пакетов), упрощающих отображение информации в различных 
сферах инженерной и научной деятельности. 

\newpage
\section{Индивидуальное задание}



\subsection{Пределы и непрерывность.}

\begin{center}
\textbf{Задача № 1.}   
\end{center}

\textbf{Условие:}

Дана последовательность  $a_{n}=\frac{4n-1}{2n+1}$ и число $c=2$. Доказать,что $\lim\limits_{x\rightarrow\infty} a_{n}=c$, а именно для кажого $\eps > 0$ найти наименьшее натуральное число $N = N(\eps)$ такое, что $|a_{n}-c|<\eps$. Заполнить таблицу:
\begin{table}[h]
    \centering
    \begin{tabular}{|c|c|c|c|}
        \hline
         $\eps$ & $0,1$ & $0,01$ & $0,001$ \\
         \hline
         $N(\eps)$ &  &  & \\
         \hline
    \end{tabular}
\end{table}

\textbf{Решение:}
$$a_{n}=\frac{4n-1}{2n+1}; \; c = 2$$ 
Найдём лимит $a_{n}$:
$$\lim\limits_{x\rightarrow\infty} a_{n}= 2 = c;$$
Рассмотрим  $|a_{n}-c|<\eps$:
$$|\frac{4n-1}{2n+1} - 2| <\eps;$$
$$|\frac{4n-1-4n-2}{2n+1}| <\eps;$$
$$\frac{3}{2n+1} < \eps;$$
$$n > \frac{3-\eps}{2\eps};$$
При $\eps = 0,1$ получим:
$$ n > \frac{3-0,1}{2*0,1} \Leftrightarrow n > 14,5$$
При $\eps = 0,01$ получим:
$$ n > \frac{3-0,01}{2*0,01} \Leftrightarrow n > 149,5$$
При $\eps = 0,001$ получим:
$$ n > \frac{3-0,001}{2*0,001} \Leftrightarrow n > 1499,5$$
Заполним таблицу:
\begin{table}[h]
    \centering
    \begin{tabular}{|c|c|c|c|}
        \hline
         $\eps$ & $0,1$ & $0,01$ & $0,001$ \\
         \hline
         $N(\eps)$ & $15$ & $150$  & $1500$ \\
         \hline
    \end{tabular}
\end{table}

\newpage
\begin{center}
\textbf{Задача № 2.}   
\end{center}

\textbf{Условие:}
Вычислить пределы функций
\begin{table}[h]
    \centering
    \begin{tabular}{c|c}
         a & $\lim\limits_{x\rightarrow 1} \frac{x^3+5x^2+3x-9}{x^3+4x^2-4x-1}$ \\
         \hline
         б & $\lim\limits_{x\rightarrow\infty} \frac{2x\sqrt{x} + \sqrt{1+9x^3}+x}{3x\sqrt{x+10}}$ \\
         \hline
         в & $\lim\limits_{x\rightarrow 2} \frac{\sqrt[3]{4x}-2}{\sqrt{2+x}-\sqrt{2x}}$\\
         \hline
         г & $\lim\limits_{x\rightarrow 1} (\frac{2x-1}{x})^{\frac{1}{\sqrt[3]{x}-1}}  $\\
         \hline
         д & $\lim\limits_{x\rightarrow 0} ({\arccos{4x}})^{\lg{\sin^2{x}}}$ \\
         \hline
         е & $\lim\limits_{x\rightarrow 1} \frac{3^{5x-3} - 3^{2x^{2}}}{tg(\pi x)}$
    \end{tabular}
\end{table}

\textbf{Решение:}

а)
$$\lim\limits_{x\rightarrow 1} \frac{x^3+5x^2+3x-9}{x^3+4x^2-4x-1}$$
Получаем неопределённость: $$[\frac{0}{0}]$$
Разложим на множители:
$$\lim\limits_{x\rightarrow 1} \frac{(x+3)^{2}(x-1)}{(x^2+5x+1)(x-1)} $$
Сокращаем одинаковые множители:
$$\lim\limits_{x\rightarrow 1} \frac{(x+3)^{2}}{(x^2+5x+1)}=\frac{16}{7}$$

б)
$$\lim\limits_{x\rightarrow\infty} \frac{2x\sqrt{x} + \sqrt{1+9x^3}+x}{3x\sqrt{x+10}}$$
Получаем неопределённость: $$[\frac{0}{0}]$$
Делим на $x^{\frac{3}{2}}$ верхнюю и нижнюю часть:
$$\lim\limits_{x\rightarrow\infty} \frac{2+\sqrt{\frac{1}{x}+9}+\frac{1}{\sqrt{x}}}{3\sqrt{1+\frac{10}{x}}} = \frac{5}{3} $$


в)
$$\lim\limits_{x\rightarrow 2} \frac{\sqrt[3]{4x}-2}{\sqrt{2+x}-\sqrt{2x}}$$
Получаем неопределённость: $$[\frac{0}{0}]$$
Домножим начальное выражение на сопряженные множители:
$$\lim\limits_{x\rightarrow 2} \frac{(4x-8)(\sqrt{2+x}+\sqrt{2x})}{(2-x)(\sqrt[3]{(4x)^{2}}+2\sqrt[3]{4x}+4)} $$
\newpage
Сократим множители и вынесем коэффициент за предел:
$$-4\lim\limits_{x\rightarrow 2} \frac{(\sqrt{2+x}+\sqrt{2x})}{(\sqrt[3]{(4x)^{2}}+2\sqrt[3]{4x}+4)} =-\frac{4}{3} $$

г)
$$\lim\limits_{x\rightarrow 1} (\frac{2x-1}{x})^{\frac{1}{\sqrt[3]{x}-1}}  $$
Подставляя значение получим:
$$[ 1^{\infty}]$$
Используя секретную формулу \textnumero $197$ получим:
$$e^{\lim\limits_{x\rightarrow 1} (\frac{2x-1}{x} - 1)({\frac{1}{\sqrt[3]{x}-1}})}$$
Считаем чему равна степень:
$$\lim\limits_{x\rightarrow 1} (\frac{2x-1}{x} - 1)({\frac{1}{\sqrt[3]{x}-1}})$$
$$\lim\limits_{x\rightarrow 1} (\frac{x-1}{x})({\frac{1}{\sqrt[3]{x}-1}})$$
Умножаем на сопряженное выражение к $\sqrt[3]{x}-1$:
$$\lim\limits_{x\rightarrow 1} \frac{\sqrt[3]{x^{2}} + \sqrt[3]{x}+1}{x}=3$$
Значит предел начального выражения равен:
$$e^{3}$$

д)
$$\lim\limits_{x\rightarrow 0} ({\arccos{4x}})^{\lg{\sin^2{x}}}$$
Подставляя значение получим:
$$[(\frac{\pi}{2})^{-\infty}]= 0$$

е)
$$\lim\limits_{x\rightarrow 1} \frac{3^{5x-3} - 3^{2x^{2}}}{tg(\pi x)}$$
Получаем неопределённость: $$[\frac{0}{0}]$$
$$\lim\limits_{x\rightarrow 1} \frac{3^{2x^{2}}(3^{5x-3-2x^{2}} - 1)}{{tg(\pi x)}}$$
Выполним замену переменных $t = x - 1;\; t \rightarrow 0$ 
$$\lim\limits_{t\rightarrow 0} \frac{3^{2(t+1)^{2}}(3^{t-2t^{2}} - 1)}{{tg(\pi t + \pi)}}$$
По формуле поворота для тангенса получим:
$$\tg(\pi t+ \pi))= tg(\pi t)$$
Заменим это выражение на эквивалетное при $t \rightarrow 0$:
$$tg(\pi t)  \sim \pi t$$

\newpage
Подставим в знаменатель это выражение:
$$\lim\limits_{t\rightarrow 0} \frac{3^{2(t+1)^{2}}(3^{t-2t^{2}} - 1)}{{\pi t }}$$
Заменим $3^{t-2t^{2}} - 1$ на эквивалентное при $t \rightarrow 0$:
$$3^{t-2t^{2}} - 1 = t(1-2t)*\ln{3}$$
Подставим в числитель это выражение:
$$\lim\limits_{t\rightarrow 0} \frac{3^{2(t+1)^{2}}(t(1-2t)*\ln{3})}{{\pi t }}$$
Сократим:
$$\lim\limits_{t\rightarrow 0} \frac{3^{2(t+1)^{2}}((1-2t)*\ln{3})}{{\pi}} = \frac{9\ln{3}}{\pi}$$ \\
Ответ:а) $\frac{16}{7}$; б)$\frac{5}{3}$; в)$-\frac{4}{3}$; г)$e^{3}$; д)$0$;
е)$\frac{9\ln{3}}{\pi}$

\newpage
\begin{center}
\textbf{Задача № 3.}   
\end{center}
\textbf{Условие:}
а) Показать, что данные функции $f(x)$ и $g(x)$ являются бесконечно малыми или бесконечно большими
при указанном стремлении аргумента. б) Для каждой функции $f(x)$ и $g(x)$ записать главную часть
(эквивалентную ей функцию)  вида $C(x-x_0)^{\alpha}$ при $x\rightarrow x_0$ или $Cx^{\alpha}$
при $x\rightarrow\infty$, указать их порядки малости (роста). в) Сравнить функции $f(x)$ и $g(x)$ при указанном стремлении.$f(x) = \sqrt[3]{1-{\sqrt[3]{x}}};\; g(x) = 4(x-1)^{2}$

\textbf{Решение:}

а)
$$\lim\limits_{x\rightarrow 1} f(x) = \lim\limits_{x\rightarrow 1}\sqrt[3]{1-{\sqrt[3]{x}}}=0 $$ 
Получается f(x) БМ.
$$\lim\limits_{x\rightarrow 1} g(x) = \lim\limits_{x\rightarrow 1} 4(x-1)^{2} = 0 $$
Получается g(x) БМ.

б)
Нужно привести к виду:
$$\underset{x \rightarrow 1}{f(x)} = C(x-1)^{p}$$
$$\underset{x \rightarrow 1}{g(x)} = C(x-1)^{p}$$
$$f(x) = \sqrt[3]{1-{\sqrt[3]{x}}}$$
Выполним замену переменных $t = x-1; \; t \rightarrow 0:$
$$f(x)=\sqrt[3]{1-{\sqrt[3]{t+1}}} \sim -\sqrt[3]{\frac{1}{3} t}$$
Получим:
$$f(x)= \sqrt[3]{\frac{1}{3}} * (x-1)^{\frac{1}{3}}$$
$f(x)$ - БМ $\frac{1}{3}$ порядка
Для $g(x)$:
$$g(x) = 4(x-1)^{2}$$
$g(x)$ - БМ $2$ порядка

в)
$$\lim\limits_{x\rightarrow 1} \frac{g(x)}{f(x)} = 0$$ 

Получается $f(x) = o(g(x))$

\newpage
\addcontentsline{toc}{section}{Список литературы}
\begin{thebibliography}{99}
\bibitem{book01} Львовский С.М. Набор и вёрстка в системе \LaTeX, 2003 c.
\bibitem{book02} Котельников И.А. \LaTeX \; 2e по-русски, 2004.
\end{thebibliography}

\end{document}